\documentclass{article}
\usepackage[utf8]{inputenc}
\usepackage{graphicx}
\usepackage{float}
\usepackage[frenchb]{babel}


\title{
  Projet Prog 2 : Tower Defense\\
  \large Compte Rendu 1
}
\author{Raphaël LE BIHAN, Rida LALI}

\begin{document}

\maketitle

\section{Présentation du projet}

\subsection{Présentation générale}

L'objectif du projet est de réaliser un jeu de Tower Defense.\\
Le jeu présente une carte 2D où est située une tour appelée tour principale, reliée au bord de l'écran par un chemin. Lors de différentes manches, des ennemis arrivent du bord de l'écran et tentent de prendre d'assaut la tour principale.\\
L'objectif du joueur est de défendre la tour principale de l'assaut ennemi. Pour cela il peut disposer différentes tours de défense sur la carte, qui attaqueront les ennemis à proximité.

\subsection{Format du rendu}

Les explications qui suivent dans ce compte rendu, (en particulier l'organisation du code, et les détails de plusieurs classes et objets) sont détaillées dans différents autres documents joins avec celui-ci.
Il n'est pas nécessaire de tout lire pour comprendre l'organisation du code source et les mécanismes du jeu.
Les parties importantes à lire dans les autres documents seront signalées dans la suite du compte rendu.

\section{Organisation du code}

Détail important : presentation.pdf

Le code est organisé en différents packages :

\begin{itemize}
\item
  jeu : mécanismes généraux pour le déroulement d'une partie (carte, magasin, ticks, définition d'un ennemi/d'une tour, etc..)
\item
  affichage : affichage d'un partie dans une fenêtre. Le package est divisé en deux sous-packages : \\
  composants (les différents composants de la fenêtre, et leur disposition) \\
  comportement (les effets des boutons, évènements, etc..)
\item
  strategies : algorithmes généraux pour les actions des ennemis et tours (recherche de cibles à attaquer/soigner, pathfinding, etc..)
\item
  ennemis, tours, effets : implémentation des différents ennemis, tours, effets
\item
  test : partie test, qui est lancée à l'exécution du programme
\end{itemize}

Dans les packages jeu et affichage, il a été choisi de ne pas implémenter les méthodes directement dans les différentes classes (abstraites ou pas), mais dans des objets (qui sont ``comme'' des objets compagnons, mais avec un nom différent parce qu'on ne peut pas séparer un objet compagnon et une classe dans deux fichiers en Scala).
Ceci a été fait afin de diminuer la taille des fichiers (comme conseillé dans le sujet, sinon certains fichiers devenaient énormes).


\section{Mécanismes : affichage et jeu}

Détail important : presentation.pdf et mecanismes.pdf

\section{Tours, ennemis, effets, manches implémentés}

Les tours et ennemis implémentés ont différents comportements. \\
Les ennemis se déplacent différemment : certains choisissent d'emprunter seulement les chemins accessibles (avec aucune tour sur le passage), d'autres prennent le chemin le plus court même s'il contient des tours, il essaieront de les détruire en avançant. \\
Les ennemis et les tours choisissent des cibles différentes, par exemple la cible la plus proche, ou la tour principale en priorité, ou la cible la plus dangereuse, etc..

L'architecture choisie (organisation en package + utilisation de l'héritage) permet d'ajouter facilement des tours, ennemis et effets supplémentaires au jeu.

\subsection{Tours}

Au total 8 tours ont été implémentées, certaines peuvent être achetées directement, d'autres sont accessibles en améliorant une tour déjà présente (voir Magasin et Contrat dans detail\_jeu.pdf).

\begin{itemize}
\item
  TourPrincipale : Non achetable, la tour principale est la tour à défendre
\item
  Defenseuse : Tour classique d'attaque, qui attaque l'ennemi le plus proche
\item
  Sniper : Evolution de Defenseuse, a une grande portée et attaque l'ennemi le plus dangereux
\item
  Mortier : Evolution de Defenseuse, attaque l'ennemi le plus proche en faisant des dégâts de zone
\item
  Lampadaire : Tour de soin, qui soigne la tour à portée ayant le moins de PV, et qui puisse être soignée
\item
  Gluant : Evolution de Lampadaire, inflige un effet de ralentissement aux ennemis à proximité
\item
  Yogi : Evolution de Lampadaire, elle est capable à la fois d'attaquer et se soigner
\item
  Barrière : Tour inoffensive mais avec un grand nombre de PV, qui permet de bloquer le passage aux ennemis
\end{itemize}

\subsection{Ennemis}

Au total 5 ennemis ont été implémentés, avec des comportements différents

\begin{itemize}
\item
  Fourmi : Ennemi peu puissant, se déplace vers la tour principale en prenant un chemin sans tour.
\item
  Racaillou : Ennemi lent mais plus puissant, prend le chemin le plus cours vers la tour principale.
\item
  Kamikaze : Ennemi rapide qui évite les tours et rush vers la tour principale pour se suicider. Il n'attaque pas les tours mais explose et inflige des dégâts aux tours et ennemis proches lorsqu'il meurt.
\item
  Soignant : Ennemi inoffensif mais qui se déplace vers les ennemis ayant subi des dégâts pour les soigner.
\item
  Johnson : C'est le boss final, avec un grand nombre de PV.
\end{itemize}

\subsection{Effets}

Seul un effet a été implémenté : Ralenti, qui divise par deux la vitesse de déplacement d'un endommageable pendant un instant.
Cet effet est infligé aux ennemis par la tour Gluant.

\subsection{Manches}

Le jeu contient pour l'instant une partie de quatre manches (trois manches classiques + le boss de fin).
Comme pour les tours, ennemis et effets l'architecture permet de changer/ajouter facilement d'autres manches.

\section{Affichage}

L'interface graphique comporte différents boutons et commandes dont l'utilisation est (voulue) assez intuitive.\\
Le comportement général est détaillé dans detail\_affichage.pdf

Le jeu comporte deux timers, TimerAff pour l'affichage (à fréquence fixe), TimerJeu (à fréquence réglable, s'arrête entre les différentes manches), afin de pouvoir régler la vitesse de déroulement du jeu.

Le magasin (Inventaire) et la zone d'affichage de la carte (ZoneGrille) héritent de la classe Selectionneur, qui permet de sélectionner une case en cliquant dans un quadrillage.

On peut utiliser la molette, ou les boutons pour zoomer, dézoomer et se déplacer sur la carte. De plus, au survol par la souris des informations sont affichées dans la zone inférieure droite de la fenêtre (ZoneInfos).

Les boutons ATTAQUE, ACHETER, DETRUIRE permettent de lancer la prochaine manche, acheter/faire évoluer ou supprimer une tour.


\section{Evolution du projet}

Certaines parties concernant l'affichage du jeu n'ont pas encore été implémentées par manque de temps, notamment :

\begin{itemize}
\item
  l'affichage des messages (ZoneGrilleM),
\item
  le drag n drop pour se déplacer sur la carte (ZoneGrilleDND),
\item
  les boutons pause et paramètres
\item
  l'affichage de messages sur toute la fenêtre (par exemple pour afficher VICTOIRE/GAME OVER ou PAUSE)
\item
  affichage des attaques des ennemis/tours sur la carte
\end{itemize}

On pourra aussi rajouter plus de tours, ennemis, effets ou tuiles avec de nouvelles mécaniques de jeu (ex: cases d'eau, tirs en ligne droite comme proposés dans l'énoncé).

Une autre fonctionnalité qui pourra être rajoutée est la lecture d'une partie (carte + magasin + manches) directement depuis un fichier au lieu de l'implémenter dans le code source.

\end{document}

